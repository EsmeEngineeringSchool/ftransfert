%%%%%%%%%%%%%%%%%%%%%%%%%%%%%%%%%%%%%%%%%%%%%%%%%%%%%%%%%%%%%%%%%%%
%  Fichier généré par gentex.py !
%%%%%%%%%%%%%%%%%%%%%%%%%%%%%%%%%%%%%%%%%%%%%%%%%%%%%%%%%%%%%%%%%%%
\documentclass{article}
\usepackage[francais]{babel}
\usepackage[utf8]{inputenc}
\usepackage[T1]{fontenc}
\usepackage{lmodern}
\usepackage{amsmath}
\usepackage{amsthm}
\usepackage{amssymb}
\usepackage{pgf,tikz}
\usepackage{pgfplots}
\usepackage{siunitx}
\usepackage{geometry}
\usepackage{mymath}
\geometry{
paperwidth=21cm,
paperheight=29.7cm,
margin=1cm
}
\usepackage{tabularx}
\newcolumntype{K}[1]{>{\centering\arraybackslash}p{#1}}
\newcolumntype{M}[1]{>{\centering\arraybackslash}m{#1}}
\newcolumntype{N}{@{}m{0pt}@{}}
\begin{document}
\thispagestyle{empty}
$$\boldsymbol{H(p)=\dfrac{100(p+1)^{2}(0.01p+1)}{(10000p+1)(100p+1)(0.0001p+1)^{3}}}$$
\begin{center}
\begin{tikzpicture}[trim axis left]
\begin{axis}
[ticklabel style = {font=\normalsize},
width=0.9\textwidth,
height=0.25\textheight,
grid=both,
major grid style={black!40},
label style={font=\large},
xmode=log,ymode=normal,ylabel={Gain (\si{\decibel})},
xtick={1e-8,1e-7,1e-6,1e-5,1e-4,1e-3,1e-2,1e-1,1e0,1e1,1e2,1e3,1e4,1e5,1e6,1e7,1e8},
ytick={-225.0,-187.5,-150.0,-112.5,-75.0,-37.5,0.0,37.5,75.0},
xticklabels={$10^{-8}$,$10^{-7}$,$10^{-6}$,$10^{-5}$,$10^{-4}$,$10^{-3}$,$10^{-2}$,$10^{-1}$,$10^{0}$,$10^{1}$,$10^{2}$,$10^{3}$,$10^{4}$,$10^{5}$,$10^{6}$,$10^{7}$,$10^{8}$},
ytick={-225.0,-187.5,-150.0,-112.5,-75.0,-37.5,0.0,37.5,75.0},
xmin=1e-8,xmax=1e8,
ymin=-225.0,ymax=75.0
]
\addplot[ultra thick, blue,domain=1e-8:1e8, samples=201] {40-10*log10(1+100000000*x*x)-10*log10(1+10000*x*x)+20*log10(1+1*x*x)+10*log10(1+0.0001*x*x)-30*log10(1+1e-08*x*x)};
\addplot[line width=2pt,red,dashed,domain=1e-8:1e-4, samples=51] {40.0+0.0*log10(x)};
\addplot[line width=2pt,red,dashed,domain=1e-4:1e-2, samples=51] {-40.0+-20.0*log10(x)};
\addplot[line width=2pt,red,dashed,domain=1e-2:1e0, samples=51] {-80.0+-40.0*log10(x)};
\addplot[line width=2pt,red,dashed,domain=1e0:1e2, samples=51] {-80.0+0.0*log10(x)};
\addplot[line width=2pt,red,dashed,domain=1e2:1e4, samples=51] {-120.0+20.0*log10(x)};
\addplot[line width=2pt,red,dashed,domain=1e4:1e8, samples=51] {120.0+-40.0*log10(x)};
\draw[draw=none,fill=blue] (axis cs:1e-05,39.95678192009956) circle (2pt);
\draw[draw=none,fill=blue] (axis cs:0.0001,36.9892658574548) circle (2pt);
\draw[draw=none,fill=blue] (axis cs:0.01,-3.0098656404547706) circle (2pt);
\draw[draw=none,fill=blue] (axis cs:1,-73.97940026043817) circle (2pt);
\draw[draw=none,fill=blue] (axis cs:100,-76.99013435956257) circle (2pt);
\draw[draw=none,fill=blue] (axis cs:10000,-49.030465510296224) circle (2pt);
\draw[draw=none,fill=blue] (axis cs:100000,-80.12963686966805) circle (2pt);
\end{axis}
\end{tikzpicture}

\begin{tikzpicture}[trim axis left]
\begin{axis}
[ticklabel style = {font=\normalsize},
width=0.9\textwidth,
height=0.25\textheight,
grid=both,
major grid style={black!40},
label style={font=\large},
xmode=log,ymode=normal,xlabel={Pulsation (\si{\radian\per\second})},
ylabel={Phase (\si{degree})},
xtick={1e-8,1e-7,1e-6,1e-5,1e-4,1e-3,1e-2,1e-1,1e0,1e1,1e2,1e3,1e4,1e5,1e6,1e7,1e8},
ytick={-218.0,-177.25,-136.5,-95.75,-55.0,-14.25,26.5,67.25,108.0},
xticklabels={$10^{-8}$,$10^{-7}$,$10^{-6}$,$10^{-5}$,$10^{-4}$,$10^{-3}$,$10^{-2}$,$10^{-1}$,$10^{0}$,$10^{1}$,$10^{2}$,$10^{3}$,$10^{4}$,$10^{5}$,$10^{6}$,$10^{7}$,$10^{8}$},
ytick={-218.0,-177.25,-136.5,-95.75,-55.0,-14.25,26.5,67.25,108.0},
xmin=1e-8,xmax=1e8,
ymin=-218.0,ymax=108.0
]
\addplot[ultra thick, blue,domain=1e-8:1e8, samples=201] {-1*atan2(10000*x,1)-1*atan2(100*x,1)+2*atan2(1*x,1)+1*atan2(0.01*x,1)-3*atan2(0.0001*x,1)};
\addplot[line width=2pt,red,dashed,domain=1e-8:1e-4, samples=51] {0.0};
\draw[line width=2pt,red,dashed] (axis cs:1e-4,0.0)  -- (axis cs:1e-4,-90.0);
\addplot[line width=2pt,red,dashed,domain=1e-4:1e-2, samples=51] {-90.0};
\draw[line width=2pt,red,dashed] (axis cs:1e-2,-90.0)  -- (axis cs:1e-2,-180.0);
\addplot[line width=2pt,red,dashed,domain=1e-2:1e0, samples=51] {-180.0};
\draw[line width=2pt,red,dashed] (axis cs:1e0,-180.0)  -- (axis cs:1e0,0.0);
\addplot[line width=2pt,red,dashed,domain=1e0:1e2, samples=51] {0.0};
\draw[line width=2pt,red,dashed] (axis cs:1e2,0.0)  -- (axis cs:1e2,90.0);
\addplot[line width=2pt,red,dashed,domain=1e2:1e4, samples=51] {90.0};
\draw[line width=2pt,red,dashed] (axis cs:1e4,90.0)  -- (axis cs:1e4,-180.0);
\addplot[line width=2pt,red,dashed,domain=1e4:1e8, samples=51] {-180.0};
\draw[draw=none,fill=blue] (axis cs:1e-05,-5.766737424633307) circle (2pt);
\draw[draw=none,fill=blue] (axis cs:0.0001,-45.56142396491293) circle (2pt);
\draw[draw=none,fill=blue] (axis cs:0.01,-133.2756262163559) circle (2pt);
\draw[draw=none,fill=blue] (axis cs:1,-88.86558176049746) circle (2pt);
\draw[draw=none,fill=blue] (axis cs:100,42.14109338529432) circle (2pt);
\draw[draw=none,fill=blue] (axis cs:10000,-45.58433998481058) circle (2pt);
\draw[draw=none,fill=blue] (axis cs:100000,-162.92665647663213) circle (2pt);
\end{axis}
\end{tikzpicture}
\end{center}
\paragraph{Fonctions réelles du gain et du déphasage}
$$G(\omega)=|H(\jw)|=\dfrac{100\left(1+\tau_3^2\omega^2\right)\sqrt{1+\tau_4^2\omega^2}}{\sqrt{1+\tau_1^2\omega^2}\sqrt{1+\tau_2^2\omega^2}\left(1+\tau_5^2\omega^2\right)^{\frac{3}{2}}}$$
$$G_{dB}(\omega)=40-10\log{(1+\tau_1^2\omega^2)}-10\log{(1+\tau_2^2\omega^2)}+20\log{(1+\tau_3^2\omega^2)}+10\log{(1+\tau_4^2\omega^2)}-30\log{(1+\tau_5^2\omega^2)}$$
$$\phi(\omega)=\arg{H(\jw)}=-\arctan{\tau_1\omega}-\arctan{\tau_2\omega}+2\arctan{\tau_3\omega}+\arctan{\tau_4\omega}-3\arctan{\tau_5\omega}$$
\paragraph{Quelques valeurs particulières (calculées) :}

\begin{center}
\resizebox{0.95\textwidth}{!}{%
\begin{tabular}{|M{3cm}|M{2cm}|M{2cm}|M{2cm}|M{2cm}|M{2cm}|M{2cm}|M{2cm}|N}
\hline
Pulsation (\si{\radian\per\second})&$10^{-5}$&$10^{-4}$&$10^{-2}$&$10^{0}$&$10^{2}$&$10^{4}$&$10^{5}$& \\[1em] 
\hline
Gain (\si{\decibel})&40&37&-3&-74&-77&-49&-80& \\[1em] 
\hline
Déphasage (\si{\degree})&-6&-46&-133&-89&42&-46&-163& \\[1em] 

    \hline
\end{tabular}%
}
\end{center}
\paragraph{Commande pour reproduire ce fichier :}
\begin{verbatim}bodePGFtikz.py --taus [(10000, -1), (100, -1), (1, 2), (0.01, 1), (0.0001, -3)]--axis estimation\end{verbatim}\end{document}
%%%%%%%%%%%%%%%%%%%%%%%%%%%%%%%%%%%%%%%%%%%%%%%%%%%%%%%%%%%%%%%%%%%
